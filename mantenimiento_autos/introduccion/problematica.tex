\section{Problemática}
En la empresa Productos Plazco S.A, presentan diversos problemas dentro del área de mantenimiento vehicular. Se registra que dentro del periodo Enero-Diciembre de 2019 se encontraron fallas en los siguientes procesos:
\begin{itemize}
	\item Mal control de inventario para las refacciones que se llegan a utilizar para el mantenimiento de los vehículos .
	\item No se lleva un registro de las entradas y salidas de los vehículos.
	\item Falta de especificación del trabajo al momento de ingresar un vehículo.
	\item Desfase en los tiempos de trabajo. No se cuenta con un protocolo para la definición de los trabajos a realizar por lo que algunas actividades tardan mas de lo deseado.
	\item Mala asignación para la realización de tareas.
	\item Mala comunicación con los conductores de cada unidad
	Todas estas fallas se deben a que no se lleva un buen control, ni una buena administración para la asignación de tareas, al igual que se detecta una falta de organización al momento de el registro de inventarios y una falla en la comunicación con los choferes de cada unidad vehicular. De la misma forma el principal problema que se detecta es al momento de ingresar las unidades al taller ya que no se define desde un inicio el trabajo que se debe realizar, el tiempo estimado y las personas responsables del mismo, esto genera un tiempo excesivo para el mantenimiento de cada vehículo y una pérdida del control en las refacciones necesarias para cada trabajo.
\end{itemize}