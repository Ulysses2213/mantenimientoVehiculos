\section{Plan de Desarrollo}
Para la planificación de este proyecto se toman en cuanta varios aspectos y elementos que se llevan a cabo a lo largo del ciclo de vida de un software. La primera de ellas es la \textbf{planificación} aquí es donde se obtienen los requerimientos y como lo mencionamos anteriormente, estos se obtienen por medio de entrevistas o cuestionarios que son aplicados a cada uno de los usuarios que estará involucrado en la interacción con el software. Básicamente, la planificación es el paso previo al inicio de cualquier proyecto de software y el más importante.
\\ 
Una vez que la planificación se ha realizado y se han recopilado todas las ideas con el cliente o los usuarios, se realiza un \textbf{análisis} desde el ámbito del desarrollo. Se hace una documentación de todo lo recabado y se muestra en uno o varios documentos, por ejemplo: documento de requerimientos, diagrama entidad - relación (base de datos), diagramas de actividades, diagramas de secuencia. 
\\ 
Una vez hecho en análisis, viene la \textbf{implementación, las pruebas y la documentación} en esta parte o 'fase' del proyecto comienza la codificación de todo el análisis que se realizó. Posteriormente, pasamos a las pruebas que se pueden hacer de caja negra (directamente con interfaz) o de caja blanca (análisis de código fuente). La documentación es algo que, desafortunadamente, casi no se toma en cuenta, pero hay que señalar que es tan importante como la planificación, en esta parte básicamente se plasma en uno o varios documentos que es lo que se ha realizado en el desarrollo del software: tecnología utilizada, tipos de datos, diagramas, diseño, implementación, etc.
\\
El \textbf{despliegue y mantenimiento } comienza cuando las pruebas han sido realizadas y el software las ha pasado todas o la mayoría (en este caso, tomaremos el 95\%), el programa y la base de datos estarán puestas en producción como un ejecutable y en la nube respectivamente. La \textbf{capacitación} es de suma importancia y algo que tampoco se toma muy en cuenta, se planea hacer un manual de usuario pero también realizar una capacitación a todo el personal ya que los usuarios normalmente están opuestos al cambio tan radical de realizar sus tareas con ayuda de un software. Por ultimo, el mantenimiento, aquí también se consideran las mejoras que se le pueden aplicar al programa desplegado algo muy importante a destacar es que normalmente esta fase toma más tiempo que el propio desarrollo del programa. Este planea realizarse cuando el cliente necesite un cambio en sus operaciones, si la tecnología utilizada evoluciona o se realizará una migración completa de tecnología. 

\subsection{Metodología SCRUM}
Sabemos que SCRUM es una metodología ági y su flujo de trabajo es, hasta cierto punto, sencillo. trabajaremos con 'sprints' el cual es un intervalo de tiempo establecido (semanal, para este caso) aunque también podremos ir ajustando de acuerdo al desempeño del equipo y a todo el tiempo que se tiene para realizar el proyecto. 
\subsubsection{Planificación del Sprint}
Uno de los principales objetivos de las diversas reuniones que se hacen es identificar y comunicar cual será la evolución del desarrollo en el actual sprint.
\subsubsection {Scrum Diario} 
Comúnmente llamado Daily StandUp, se hará una pequeña reunión diaria para lograr que todo el equipo de desarrollo se mantenga actualizada con toda la carga de trabajo que se tiene planeada para ese día y apegados el sprint actual. Para esta pequeña reunión, se tienen estas 'reglas'. 
\begin{itemize}
	\item La reunión debe durar entre 5 a 10 minutos. 
	\item Reunir a todo el equipo en un lugar adecuado, donde el espacio sea suficiente y de preferencia material didáctico para expresar ideas. 
	\item Hacerla de pie, para una mayor concentración y agilidad.
	\item El mismo horario y lugar SIEMPRE. 
\end{itemize}
\subsubsection{Revisión y Retrospectiva del Sprint}
Al final de un sprint, necesitamos hacer una revisión de este mismo aquí se presentan los trabajos completados (evolución del proyecto) además de que se realizará una retrospectiva del propio sprint en la cual cada uno de los miembros del equipo plasmarán lecciones aprendidas. El objetivo es implementar una mejora continua. Cabe señalar que esta revisión tendrá una duración de 4 horas. 
\subsubsection{Documentos}
\begin{itemize}
	\item \textbf{Product Backlog: }documento donde se plasmarán todos los requerimientos del proyecto y se hace una priorización de cada uno de estos (Alta, Media, Baja). Representa todo aquello que se construirá. 
	\item \textbf{Sprint Backlog: } documento que contentrá una especie de 'subrequerimientos' o actividades que se deben de realizar. Aquí se realizará lo siguiente: 
	\begin{itemize}
		\item Asignación de horas de trabajo a cada actividad.
		\item Si una tarea dura más de 16 horas, esta se va a subdividir en otras más pequeñas. 
		\item Estas actividades nunca serán asignadas, el equipo de desarrollo las tomará a voluntad.
	\end{itemize} 
\end{itemize}
