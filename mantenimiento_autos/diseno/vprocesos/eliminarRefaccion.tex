\subsubsection{Eliminar Refaccion}
En la siguiente imagen \ref{fig:Diagrama de Secuencia - Eliminar Refaccion}, se muestra el diagrama de secuencia correspondiente a la eliminación de algún registro de una refacción que estaba en alamcén del taller, esto en el caso de que ya no haya en existencia o el proveedor no las haya conseguido. El sistema muestra un 'mensaje de seguridad' para verificar al administrador si esta seguro de borrar ese registro de la base de datos. Existen dos posibilidades dentro de este módulo:
\begin{itemize}
	\item \textbf{Aceptación:} El administrador acepta que desea eliminar ese registro, el sistema solicita a la base de datos la eliminación de dicho registro y se muestra un mensaje de que el empleado ha sido eliminado de manera satisfactoria.
	\item \textbf{Cancelación:} Si elige la opción 'Cancelar' en la interfaz de usuario, el sistema desaparece el 'mensaje de seguridad' y la base de datos queda intacta. 
\end{itemize}
\begin{figure}[!h]
	\centering
	\includegraphics[width=0.8\textwidth]{./diseno/vprocesos/imagenes/eliminarRefaccion}
	\caption{Diagrama de Secuencia - Eliminar Refacción}
	\label{fig:Diagrama de Secuencia - Eliminar Refaccion}
\end{figure}
\clearpage