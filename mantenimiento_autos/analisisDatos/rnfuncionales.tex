\section{Requerimientos No Funcionales}
Estos requerimientos no intervienen en la funcionalidad del sistema, sin embargo, es importante tomarlos en cuenta ya que se expresan algunas características fundamentales del propio sistema, de igual forma por motivos de identificación se implementa una nomenclatura RNF'X', siendo 'X' el número de requerimiento descrito:
\begin{itemize}
	\item \textbf{RNF1:}La interfaz gráfica de usuario (GUI) debe de estar bien diseñada en cada una de la pantallas.
	\item \textbf{RNF2:}Será desarrollado en el lenguaje de programación Java con ayuda del IDE Apache Netbeans en su versión 12.4. Esto para aprovechar que el lenguaje soporta multiplataforma además de que la curva de aprendizaje ya esta dada. 
	\item \textbf{RNF3:} La base de datos será desarrollada en MySQL y será montada en el servidor un servidor local con ayuda de XAMPP en su versión 3.2.4 para las pruebas del MVP. 
\end{itemize}