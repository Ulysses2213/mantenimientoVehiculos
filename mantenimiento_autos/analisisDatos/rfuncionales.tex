\section{Requerimientos Funcionales}
Al analizar la problemática que presenta el cliente, llegamos a un listado de los siguientes requerimientos funcionales del proyecto, cabe señalar para la correcta identificación de estos requerimientos, se implementa la nomenclatura RF'X', siendo 'X' el número de requerimiento descrito: 
\begin{itemize}
	\item \textbf{RF1:}El sistema permitirá el acceso si y solo si se ingresan de manera correcta las credenciales solicitadas, además de que exista el usuario en la base de datos.
	\item \textbf{RF2:} El sistema desplegará al empleado un menú donde podrá elegir entre la gestión de la agenda de vehículos o las refacciones.
	\item \textbf{RF3:} El sistema desplegará al administrador donde podrá elegir alguna de estas opciones: gestión de empleados, gestión de vehículos, gestión de refacciones y gestión de solicitudes.
	\item \textbf{RF4:} El sistema mostrará al empleado en una pantalla con todos aquellos registros de vehículos que el usuario tenga relacionados.
	\item \textbf{RF5:} El sistema mostrará al administrador una pantalla con todos aquellos registros ya sea de empleados, vehículos, refacciones o solicitudes. Esto dependerá de la opción que elija en el menú. 
	\item \textbf{RF6:} El sistema permitirá al empleado guardar, modificar, eliminar y buscar un registro por medio de formularios (vehículos). 
	\item \textbf{RF7:} El sistema permitirá al administrador guardar, modificar, eliminar y buscar algún registro por medio de formularios (empleados, vehículos, refacciones, solicitudes).
	\item \textbf{RF8:} El sistema validará todas y cada una de las entradas de datos, en caso de que estas sean erróneas, se mostrará un mensaje de alerta. 
	\item \textbf{RF9:} El sistema mostrará al empleado en pantalla todos aquellos registros de refacciones que haya en existencia dentro del almacén del taller. 
	\item \textbf{RF10:} El sistema permitirá al empleado solicitar una refacción por medio de un formulario de registro. 
\end{itemize}