\section{Reporte de Actividades}
\subsubsection{Productos Plazco S.A --- Reporte de Actividades \centering} 
\textbf{Folio:} RA1\\ 

\textbf{Fecha:} Correspondiente al periodo del 20 de Septiembre al 20 de Diciembre de 2020\\

\textbf{Nombre del personal responsable:} Núñez Hernández Ulises e Ibarra Ferrer Eliot Ramón.\\ 

\textbf{Departamento o proyecto:} SOFTWARE PARA APOYAR A LA LOGÍSTICA DEL MANTENIMIENTO DE AUTOMÓVILES\\ 

\textbf{Actividades desarrolladas:}
\begin{itemize}
	\item Planificación del proyecto.
	\item Análisis del sistema y requisitos:  Se extraen los requisitos del producto de software que se va a desarrollar para plasmarlos en el documento ERS (Especificación de Requerimientos del Sistema).
	\item Programación e implementación: Se realizaron los distintos procesos y estructuras que se definieron para el sistema.   
\end{itemize}

\subsubsection{Productos Plazco S.A --- Reporte de Actividades \centering}
\textbf{Folio:} RA2\\ 

\textbf{Fecha:} Correspondiente al periodo del 20 de Septiembre al 20 de Diciembre de 2020\\

\textbf{Nombre del personal responsable:} Núñez Hernández Ulises \\ 

\textbf{Departamento o proyecto:} SOFTWARE PARA APOYAR A LA LOGÍSTICA DEL MANTENIMIENTO DE AUTOMÓVILES\\ 

\textbf{Actividades desarrolladas:}
\begin{itemize}
	\item \textbf{Pruebas y revisión: }Se desarrollaron pruebas para comprobar que el software funciona correctamente con las tareas indicadas para así de esta forma asegurar la calidad del sistema.
	\item \textbf{Documentación:} Cada proceso y evolución del proyecto se han registrado desde la planificación, documentación de los requerimientos, el análisis, diseño e implementación (product backlog y sprint backlog). 
	\item \textbf{Diseño y Arquitectura del Software: }Se determinó cómo funcionará el software de forma general. Se realizaron consideraciones sobre la red, el hardware, los casos de uso, etc. La arquitectura representa la primera decisión de diseño sobre el sistema y es uno de los puntos más importantes en el proceso de desarrollo.
\end{itemize}
\clearpage



