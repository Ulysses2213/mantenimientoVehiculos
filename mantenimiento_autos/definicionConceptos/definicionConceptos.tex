\subsection{Definción de Conceptos y Productos}
Para este caso de aplicación, se han formado una lista de todos aquellos conceptos y productos que se utilizarán a lo largo del análisis, diseño e implementación del software.
\begin{itemize}
	\item \textbf{Software:} Programa informático diseñado y distribuido para satisfacer las necesidades de un grupo de personas físicas o morales, enfocado a la automatización de procesos.
	\item \textbf{Requerimiento:} Necesidad que debe de cubrir un software, normalmente estas necesidades son halladas por medio de una serie de herramientas como entrevistas o cuestionarios.
	\item \textbf{Ciclo de vida del Software:} Todos aquellos procesos que se deben de seguir para poder analizar, diseñar y construir un software desde cero o con un antecedente.
	\item \textbf{Diagrama:} Representación gráfica de una serie de procesos, normalmente son utilizados para dar a entender las diversas tareas que se llevan a cabo tanto por un humano como por una máquina.
	\item \textbf{Historia de Usuario:} Todas aquellas tareas y/o acciones que una persona (un usuario de software o de cualquier otro sistema) realiza para cumplir con un objetivo en específico. 
	\item \textbf{Vistas:} Maquetación o ventanas prediseñadas de un software.
	\item \textbf{Base de Datos:} Es un programa informático, normalmente alojado en la nube que nos permite almacenar datos de cualquier índole con un formato específico, esto se logra por medio de unas tablas que a su vez están relacionadas entre sí (Entidad - Relación). Existen las bases de datos no relacionales que no utilizan una estructura en forma de tablas si no ficheros (JSON, XML). 
	\item \textbf{Java: } Lenguaje de programación muy popular. Creado en los 90's sigue siento una tecnología muy utilizada y robusta. Utiliza el paradigma orientado a objetos. 
	\item \textbf{MySQL:} Sistema para la Gestión de Bases de Datos licenciada por Oracle, se considera como la base de datos más popular de todo el mundo.
	\item \textbf{Apache NetBeans: } Programa informático que nos permite editar el código Java para su posterior compilación y prueba.
	\item \textbf{Pruebas:} Básicamente es comprobar que el software marche bien y que existe el menor número de errores posible.
	\item \textbf{SCRUM:} Metodología ágil que nos permite la construcción 'rápida' y eficiente de un software.
	\item \textbf{Sprint Backlog:} equeño sistema de gestión de las diversas reuniones que se tienen a lo largo de todo el proyecto.
	\item \textbf{Product Backlog:} Pequeño sistema de gestión de las diversas entregas que se realizan a lo largo de la construcción del software. Se describe a detalle cada uno de los 'productos' (módulos del software) que se van a entregar o si es que existe alguna modificación. Lo más fácil y viable es utilizar una hoja de cálculo para ello.
\end{itemize}