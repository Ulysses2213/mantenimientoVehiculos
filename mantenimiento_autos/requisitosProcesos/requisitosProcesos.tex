\subsection{Requisitos de Procesos (MoProSoft)}
\subsubsection{Procesos de Operación - Administración de Proyectos Específicos.}
\begin{itemize}
	\item \textbf{Proceso:} Administración de Proyectos Específicos.
	\item \textbf{Categoría:} Operación
	\item \textbf{Propósito:} Establecer y llevar a cabo sistemáticamente las actividades que permitan cumplir con los objetivos del proyecto en tiempo y costo esperados.  
	\item \textbf{Objetivo:} Planificar el proyecto de acuerdo a los recursos que se tienen disponibles. 
	\item \textbf{Actividades:}
	\begin{itemize}
		\item Realizar un cronograma de actividades dentro de un intervalo de tiempo razonable.
		\item Calcular las horas-hombre (trabajo).
		\item Calcular los diversos costos que habrá apegado al tiempo que se invertirá y sobre todo al personal que se contratará para realizar el proyecto. 
		\item Documentar la planificación que se hará de todo el desarrollo del proyecto.
		\item Documentar cada uno de los avances y reuniones que se tengan en función a la evolución del proyecto.
	\end{itemize} 
\end{itemize}
\subsubsection{Procesos de Operación - Desarrollo y Mantenimiento de Software.}
\begin{itemize}
	\item \textbf{Proceso:} Desarrollo y Mantenimiento de Software.
	\item \textbf{Categoría:} Operación.
	\item \textbf{Propósito:} Realización sistemática de las actividades de obtención de requerimientos, análisis, diseño y construcción, integración y pruebas de productos de software nuevos o modificados cumpliendo con los requisitos especificados.  
	\item \textbf{Objetivo:} Desarrollar y mantener un software nuevo o modificado.
	\item \textbf{Actividades:} 
	\begin{itemize}
		\item Identificación del problema o problemas que se deben de resolver con ayuda del software que se desea implementar.
		\item Identificar los requerimientos tanto funcionales como no funcionales por medio de diversas herramientas como entrevistas, cuestionarios o formularios aplicados a los empleados de la organización.
		\item Documentar los requerimientos que se han obtenido para su posterior comparación con el producto final.
		\item Una vez que se tengan los requerimientos sean establecidos, se realizará una evaluación de cada uno de ellos.
		\item Realizar el análisis de acuerdo a todos los requerimientos ya clasificados, es decir, separados por funcionales y no funcionales.
		\item Verificar la viabilidad del proyecto.
		\item La fase de diseño es una de las mas importantes, aquí se realizan diversos diagramas que nos muestras las actividades que el software hará y la interacción que el usuario hará con la interfaz gráfica del programa.
		\item La construcción del sistema debe de hacerse con una metodología, en este caso utilizaremos SCRUM, haciendo entregas semanales y reuniones para ver la evolución del desarrollo. 
		\item A pesar de que la curva de conocimiento ya esta dada, no descartamos investigaciones rápidas para la codificación de los diferentes módulos a implementar.
		\item Otra fase muy importante en el desarrollo son las pruebas, estas serán de caja negra y caja blanca aplicando casos de prueba diseñados para evaluar si el sistema responde correctamente.
		\item Desarrollar un manual de usuario para que todo el personal (inclusive el nuevo) pueda interactuar con el sistema de una manera sencilla.
		\item Desarrollar un manual de operación y/o mantenimiento para su posterior mejora o modificación de algún módulo o del sistema completo. 
	\end{itemize}
\end{itemize}
